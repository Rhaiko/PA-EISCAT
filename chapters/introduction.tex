%----------------------------------------------------------------------------------------
%	INTRODUCTION.
%----------------------------------------------------------------------------------------
%example MISC \cite{exampleMISC}.\\
%example Book \cite{exampleBOOK}.\\
%example article \cite[p.~5]{exampleARTICLE}

\section{Introduction}
The purpose of this report is to relate space weather events with analyzed data from EISCAT.
The end result of this practical is to become acquainted with data from EISCAT and the programs used to analyze it.
This report will be analyzing data from the EISCAT radar through the use of GUISDAP. 
The pre-analyzed data comes from EISCAT through Madrigal database.
With GUISDAP and Madrigal, the plots will be interpreted to find relationships with space weather. 

\subsection{Space Weather}
The effect from the sun and the other extraplanetary sources on the Earth is called space weather. 
The influence of space weather on Earth include the aurora, disturbances in magnetosphere and ionosphere, and geomagnetic induced currents \cite{I_NOAA_2}. 
The causes of the space weather originates from the sun which includes coronal mass ejections, solar flares, and coronal holes.
There are several effects originating from space weather: damaging spacecraft electronics, increased drag affecting orbit of satellites, increasing radiation dosage in astronauts and airplane passengers, causing radio blackouts on Earth, and causing electrical power grid power outages.
Space weather can be observed by ground-based systems or by satellites. Ground-based systems involve telescopes observing the photosphere, neutron detectors, and measuring the ionosphere.
Many current satellites carry sun monitoring sensors as secondary payloads.
There are mathematical models that can simulate the space weather environment.

\subsection{Solar Storms of Halloween 2003}
The space weather event  occurred between October 19$^{th}$ and November 7$^{th}$ and is referred to as the Halloween solar storms. 
It occurred during the later part of solar cycle 23. It was not expected during the quieter part of the solar cycle.
It had a large affect on Earth, with aurora seen in Florida, satellites and communications being affected, and a power outage is Sweden lasting for a hour \cite{I_NASA_1}. 
It was composed of 17 flares ranging in strength, with the largest being estimated at X28 power causing a R5 radio blackout and numerous anomalies in satellites. 
This space weather event included the 4$^{th}$ largest flare in recorded history \cite{I_NOAA}.
