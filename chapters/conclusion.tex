%----------------------------------------------------------------------------------------
%	CONCLUSION.
%----------------------------------------------------------------------------------------

\section{Conclusion}
%example MISC \cite{exampleMISC}.\\
%example Book \cite{exampleBOOK}.\\
%example article \cite[p.~5]{exampleARTICLE}

This report looked at the effects of the solar flares from Halloween 2003 on the ionosphere. 
As well as into the data provided by EISCAT and Madrigad, to draw conclusions about the effects of solar flares on ionosphere.
The effects of the flare were very typical and was expected. The effects of the solar flare were clearly presented in the figures shown in the report. The unexpected consequence of the flare was a lower ion temperature, this was possibly due to the interactions of colder plasma in the foreground.
The effects of the storm was consequential but not severe in terms of health or technology impacts. This storm should serve as warning as the potential of a larger solar flare that could irreparably damage unprotected power grids or satellites. Damages to these systems could move human progress back decades.
