%\section{Discussion}
% Solar Flares

% Higher e- density (Eiscat vs IRI) 7vs0.5 peaks
% Peak at 500km instead of 350km (Eiscat vs IRI)
%%% A discussion of what phenomena you can identify in the data

\section{Discussion}

The effects of solar flares are easily observed in Figure \ref{fig:madrigal}.
The flare was at 20:38 on October 30, 2003. There is a strong increase in the electron density, a decrease in electron temperature, fluctuations in the ion temperatures, and changes in ion drift values. These values were expected and clearly shown in the data. 

The electron density disturbance peaked approximately 15 min after the solar flare and lasted for approximately an hour afterwards. This is as expected as mentioned by Handzo \cite{Handzo}, which looked at X-class solar flares effects on the atmosphere and saw strong electron density peaked after the flare with the disturbance lasting a long time . After this time period, the electron density suddenly entered a depleted state with a lower electron density than prior to the flare. This large difference between the actual electron density and the IRI-model estimate is shown in Figure \ref{fig::IRIvsMa}. The electron density during the storm is approximately a magnitude larger than the approximated model. The difference is particularly pronounced in the higher altitudes above 300 km.

The electron temperature is shown as decreased during and after the solar flare above 300km. This is in contrast to the accepted notion of highest electron temperatures during the strongest solar storms \cite{Caspi}. This discrepancy between the data shown and the expected results of the higher temperatures could be explained with differentiated plasma. This concept is explained in Battaglia \cite{Battaglia}, where there is inner cool core and the heated outer regions. Battaglia mentioned this could also be due to the cooler plasma in the foreground interacting with the higher temperature plasma. This cooler electron temperature perturbation started at the flare and lasted over 2.5 hrs. The temperature than rapidly increase as expected with a small flare at around 23:00.

During a solar flare, the ion temperatures can fluctuate during the solar storm due to positive and negative effects \cite{ITemp}. The ion temperature and the ion density interact causing an unpredictable variation of temperature. The temperature for the case of this Halloween storm fluctuated but remained higher than the preflare conditions during and after the peak. However, the highest ion temperatures were recorded half a hour before the flare, and the largest temperature variations started after the electron density entered the depleted state.  

The solar flare has a positive and negative effect on ExB ion drift velocities \cite{IDrift}. There is a short-lasting severe decrease in velocities with a longer lasting, much weaker increase in ion velocities. There was a strong negative ion drift velocity between the start and the peaking of the electron density. During and throughout the highest electron density time period, the drift velocity was the closest to zero. After the electron density enter the depleted state, there is a short period of positive drift velocities followed by short period of intense negative drift velocity before returning to preflare state.


