%----------------------------------------------------------------------------------------
%	DISCUSSION.
%----------------------------------------------------------------------------------------

\section{Discussion}
%example MISC \cite{exampleMISC}.\\
%example Book \cite{exampleBOOK}.\\
%example article \cite[p.~5]{exampleARTICLE}

The effects of solar flares are easily observed in Figure \ref{fig:madrigal}.
There is a strong increase in the electron density, a decrease in electron temperature, fluctuations in the ion temperatures, and changes in ion drift values.
Handzo \cite{Handzo} looked at X-class flares effects in the atmosphere and saw strongly increased electron density peaking after the the flare and the perturbation lasting a long time. This large difference between the actual electron density and the IRI-model estimate is shown in Figure \ref{fig::IRIvsMa}. The electron density during the storm is approximately a magnitude larger than the approximated model. The disturbance peaked after the solar flare and lasted for approximately an hour afterwards. After this time period, the electron density suddenly enters a depleted state with a lower electron density than prior to the flare.
The electron temperature is shown as decreased during and after the solar flare. This is contrary to the accepted notion of highest electron temperatures during the strongest solar storms \cite{Caspi}. This discrepancy between the the data shown and the expected results of the higher temperatures could be explained with differentiated plasma. This concept is explained in Battaglia \cite{Battaglia}, where there is cool core and the heated outer regions. Battaglia mentioned this could also be due to the cooler plasma in the foreground interacting with the higher temperature plasma.
During a solar flare, the ion temperatures can fluctuate during the solar storm due to positive and negative effects \cite{ITemp}. The ion temperature and the ion density interact interact causing an unpredictable variations. The temperature for the case of this Halloween storm fluctuated but remained higher than normal.
The solar flare has a positive and negative effect on $E \times B$ ion drift velocities \cite{IDrift}. There is a short lasting severe decrease in velocities and a longer lasting, much weaker increase in ion velocities.
